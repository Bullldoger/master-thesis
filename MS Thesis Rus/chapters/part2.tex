\chapter*{Решение задачи распространения трещины ГРП}

Гидравлический разрыв пласта (ГРП) обычно используется для повышения продуктивности залежей углеводородов и является ключевым методом доступа к нетрадиционным сланцевым коллекторам. ГРП используется не только для создания макро-масштабных трещин, но также в соединении и реактивации ранее существовавших трещин на всех масштабах для создания транспортных путей внутри пластов. Транспортная связь между существовавшими ранее разломами (то есть перфорациями, швами и естественными трещинами) часто имеет решающее значение для добычи углеводородов из пластов. Эти естественные разрушения или другие дефекты могут вызывать сложные геометрии разрушения и пути потока, возникающие как в результате разрушения при растяжении, так и в результате сдвига. 

Поскольку большинство скважин обсажены и требуют перфорирования для доступа к пласту, большинство операций по ГРП проводится через перфорацию. Перфорации играют жизненно важную роль в происхождении сложной геометрии разрушения. В трещиноватых коллекторах перфорации часто являются единственным гидравлическим соединением между трещиноватым коллектором и стволом скважины. 

Гидравлические трещины, вызванные существующими перфорационными отверстиями, будут переориентироваться в направлении нормали к минимальному напряжению в дальней зоне, когда они распространяются из ствола скважины. Вблизи ствола скважины могут возникать множественные субпараллельные трещины вследствие процессов инициирования трещин, таких как перфорации и естественные дефекты, расположенные на стенке ствола скважины с открытым стволом. Гидравлические трещины обычно возникают либо из-за перфораций, расположенных в обсаженном стволе скважины, либо из существующих ранее естественных трещин, пересекающих необсаженный ствол скважины. 

\newpage

Однако, когда угол перфорации и разность горизонтальных напряжений велики, могут возникать двойные трещины вдоль перфораций и направления максимального напряжения. Это означает, что трещины, образованные гидравлическим разрывом от ориентированных перфораций, могут быть очень сложными.

В работаx \cite{selvadurai1996mechanics}, \cite{wang2017theory}, \cite{merxhani2016introduction} представлен подробный вывод всех уравнений необходимых уравнений. Более подробное описание процессов, таких как течение суспензии в поровых средах и моделирование процесса ГРП можно найти здесь \cite{liu2019fluid}, \cite{veatch2017essentials}.

\section*{Моделирование пороупругой среды}

Система уравнений, описывающая процесс фильтрации флюида в пороупругой среде:
\begin{equation}
	\begin{multlined}
		- \nabla \cdot \sigma - \alpha \nabla p = 0 \\[10pt]
		\phi \rho_f c_t \dfrac{\partial p}{\partial t} + \nabla \cdot \left ( -\dfrac{\rho_f k}{\mu} \nabla p \right ) = 0
	\end{multlined}
\end{equation}

В представленной системе уравнений не рассматриваются процессы утечки флюида в пласт. Также исключена диффузионная составляющая из уравнения состояния для скелета:
\begin{equation*}
	- \nabla \cdot \sigma - \alpha \nabla p = \rho_s \dfrac{\partial^2 u}{\partial t^2} = 0
\end{equation*}

Объемные деформации также считаются пренебрежимо малыми:
\begin{equation*}
	\dfrac{\partial \epsilon_v}{\partial t} = \dfrac{\partial \nabla \cdot u}{\partial t} = 0
\end{equation*}

Итоговая система уравнений для дальнейшего анализа выглядит следующим образом:
\begin{equation}
	\label{eq:poroelasticity}
	\begin{multlined}
		- \nabla \cdot \sigma - \alpha \nabla p = 0, \quad \sigma = \lambda \left ( \nabla \cdot u \right ) I + 2 \mu \epsilon, \quad \epsilon = \dfrac{1}{2} \left [ \nabla u + \left ( \nabla u\right )^T \right ] \\[10pt]
		\phi \rho_f c_t \dfrac{\partial p}{\partial t} + \nabla \cdot \left ( -\dfrac{\rho_f k}{\mu} \nabla p \right ) = 0\\[10pt] \phi = \phi_0 e^{c_s \left ( p - p_0 \right )}, \quad c_t = c_f + \dfrac{\phi_0}{\phi} c_s, \quad \rho_f = \rho_0 \left ( 1 + c_f \left ( p - p_0  \right ) \right )
	\end{multlined}
\end{equation}

\newpage

Все приближения были взяты с точностью до $o(h)$ - линейного члена, так как рассматривается задача малых деформаций, поровое давление и плотности флюида меняются не значительно. При выводе системы уравнений использовался закон Дарси для описания скорости фильтрации флюида. Граничные и начальные условия будут рассмотрены далее.

% \section*{Критерий разрушения}

Рассмотрим разрушение в некоторой области, которая подчиняется линейному закону:
\begin{equation}
	\label{eq:linear_elasticity}
	-\nabla \sigma = f
\end{equation}

Пусть имеется прямоугольная область $\Omega = [0, 1] \times [0, 1]$ и граничные условия:
\begin{equation*}
	\begin{cases}
		\vec{u} = \left [ u_x, u_y \right ]^T \cdot \vec{n}  = u_{0}, \quad \vec{x} = \left [ x, y \right ] \in \partial \Omega_d \\
		\sigma \cdot n = T, \quad \vec{x} = \left [ x, y \right ] \in \partial \Omega_t
	\end{cases}
\end{equation*}

На части границы заданы фиксированные перемещения, на оставшейся границе постоянные напряжения.

Решение уравнения \eqref{eq:linear_elasticity} - поле перемещений для всей области $\Omega$.

\subsubsection{Пример}
Решение уравнения \eqref{eq:linear_elasticity} для граничных условий:
\begin{equation}
	\label{eq:boundary_conditions_1}
	\begin{cases}
		u_x = 0, \quad x = 0 \\
		u_y = 0, \quad y = 2 \\
		\sigma \cdot n = T_1 = 10^6 \text{ Па}, x = 2 \\
		\sigma \cdot n = T_2 = 5 \cdot 10^6 \text{ Па}, y = 0
	\end{cases}
\end{equation}

\begin{figure}
	\centering
	\includegraphics[width=0.65 \textwidth]{images/part2/solution_displacement_1.png}
	\caption{Решение задачи \eqref{eq:linear_elasticity} совместно с граничными условиями \eqref{eq:boundary_conditions_1}}
	\label{fig:solution_1}
\end{figure}

Данная задача решается методом конечного элемента в вариационной постановке \cite{finlayson2013method}, \cite{fletcher2012computational}:
\begin{equation}
	\label{eq:variation_linear_elasticity}
	\begin{multlined}
		\int_{\Omega} \left [ -\nabla \sigma \cdot v \right ] d \Omega = \int_{\Omega} f \cdot v d \Omega \\
		\text{Первая формула Грина \cite{фихтенгольц1968основы}:} \int_{\Omega} \left [ -\nabla \sigma \cdot v \right ] d \Omega = \int_{\Omega} \left [ \sigma \times \nabla v \right ] d \Omega - \int_{\partial \Omega} \left [ v \cdot \left ( \sigma \cdot \vec{n} \right ) \right ] d \partial \Omega \\
		\int_{\Omega} \left [ \sigma \times \nabla v \right ] d \Omega = \int_{\partial \Omega} \left [ v \cdot \left ( \sigma \cdot \vec{n} \right ) \right ] d \partial \Omega + \int_{\Omega} f \cdot v d \Omega
	\end{multlined}
\end{equation}
Домножив исходное уравнение на тестовую (пробную функцию $v$) и проинтегрировав по области получаем постановку задачи в слабой форме. Данная задача решалась с использованием свободно распространяемого пакета Fenics\footnote{Проект FEniCS представляет собой набор бесплатных программных компонентов с открытым исходным кодом, общая цель которых - обеспечить автоматическое решение дифференциальных уравнений}. Далее это программное обеспечение будет рассмотрено подробнее, важно что Fenics предоставляет инструменты для решения вариационных задач (линейных и нелинейных).
На \ref{fig:solution_1} изображено поле перемещений - решение задачи \eqref{eq:linear_elasticity}.

\subsection*{Моделирование разрушения}

Рассмотрим диаграмму деформирования\footnote{Диаграмма деформирования — графическое изображение зависимости между напряжениями и деформациями материала. Эта характеристика различна для различных материалов и определяется с помощью регистрации величины деформации при определённых приращениях величины растягивающих или сжимающих усилий}. Для построения диграммы использовалось уравнение Рамберга-Осгуда \footnote{Уравнение Рамберга-Осгуда было создано, чтобы описать нелинейную зависимость между напряжением и деформацией} :

\begin{figure}
	\centering
	\includegraphics[width=0.65 \textwidth]{images/part2/stress_strain_1.png}
	\caption{Диаграмма деформирования}
	\label{fig:stress_strain_1}
\end{figure}

Данная кривая была смоделирована и состоит из двух зон:
\begin{itemize}
	\item Зона упругих деформаций
	\item Зона пластических деформаций
\end{itemize}

При рассмотрении хрупких тел, пластические деформации, или необратимые деформации, отсутствуют, вместо этого сразу происходит разрушение. В данной работе рассматриваются только хрупкие тела и хрупкие разрушения, отсюда следует что при переходе из зоны упругих деформаций в зону пластических деформаций наступает разрушение. 

\begin{figure}
	\centering
	\includegraphics[width=0.65 \textwidth]{images/part2/linear_zone.png}
	\caption{Зона упругих деформаций}
	\label{fig:linear_zone}
\end{figure}

\newpage
На \ref{fig:linear_zone} видно, что до некоторого напряжения, тело подчиняется линейному закону Гука. Простым критерием разрушения может быть максимальное растяжение (или сжатие), при которых упругая зона переходит в пластическую, а поскольку рассматриваются только хрупкие тела, то происходит сразу разрушение, соответственно.

Для приведенной кривой максимальная деформация, которая ведет к разрушению может быть взята $\approx 5 \cdot 10^{-5}$ (\ref{fig:criterion})

\begin{figure}
	\centering
	\includegraphics[width=0.65 \textwidth]{images/part2/criterion.png}
	\caption{Критерий разрушения}
	\label{fig:criterion}
\end{figure}

Таким образом, имея задачу \eqref{eq:linear_elasticity} и критерий разрушения можно смоделировать область с заданными граничными условиями, а также предположить место разрушения.

\subsubsection{Пример}

Рассмотрим задачу \eqref{eq:linear_elasticity} и граничные условия:
\begin{equation}
	\label{eq:boundary_conditions_2}
	\begin{cases}
		u_x = u_y = 0, \quad y = 2 \\
		\sigma \cdot n = T_1 = 10^6 \text{ Па}, x = \{0, 2\} \\
		\sigma \cdot n = T_2 = 5 \cdot 10^6 \text{ Па}, y = 0
	\end{cases}
\end{equation}

Результат решения задачи - поле напряжений и объемных деформаций. Также, из поля деформаций, применительно к данной задаче, можно сделать выводы о возникновении разрушения.


\begin{figure}
	\centering
	\includegraphics[width=0.65 \textwidth]{images/part2/sigma_xx_1.png}
	\caption{Поле напряжений, $\sigma_{xx}$}
	\label{fig:sigma_xx_1}
\end{figure}

\begin{figure}
	\centering
	\includegraphics[width=0.65 \textwidth]{images/part2/sigma_yy_1.png}
	\caption{Поле напряжений, $\sigma_{yy}$}
	\label{fig:sigma_y_1}
\end{figure}

\begin{figure}
	\centering
	\includegraphics[width=0.65 \textwidth]{images/part2/sigma_yy_1.png}
	\caption{Поле объемных деформаций, $\epsilon_v$}
	\label{fig:epsilon_v}
\end{figure}

Видно, что в точках $(x = 0, y = 2)$ и $(x = 2, y = 2)$ возникают наибольшие напряжения и величина объемных деформаций также высока, отсюда можно сделать вывод, что разрушение начнется из этих точек и далее будет распространяться вдоль кривых на \ref{fig:epsilon_v}.

\subsection*{Моделирование распространения трещины ГРП}
Для решения задачи \eqref{eq:poroelasticity} требуется построить слабую форму уравнения. Подробный вывод представлен, например, в \cite{osti_1592387}.  В \eqref{eq:poroelasticity} искомыми функциями являются $(u, p) \in V$ - пространство дважды непрерывных функций. Аналогично \eqref{eq:variation_linear_elasticity} введем проверочные функции $(v, q) \in V_h $ и применим первую формулу Грина. Для первого уравнения:

\begin{equation*}
	\begin{multlined}
		\int_{\Omega} \left [ - \nabla \cdot \sigma \cdot v - \alpha \nabla p \cdot v \right ] d \Omega = \\
		= \int_{\partial \Omega} \left [ v \cdot \left ( \sigma \cdot \vec{n} \right ) \right ] d \partial \Omega + \int_{\Omega} \left [ \sigma \times \nabla v \right ] d \Omega - \int_{\partial \Omega} \alpha \nabla p \cdot v d \Omega
	\end{multlined}
\end{equation*}

Для второго уравнения вывод будет сделан в 2 
этапа, первый - замена производных по времени на дискретный аналог, явный, а далее интегрирование с пробной функцией:
\begin{equation*}
	\begin{multlined}
		\phi \rho_f c_t \dfrac{\partial p}{\partial t} + \nabla \cdot \left ( -\dfrac{\rho_f k}{\mu} \nabla p \right ) \approx \phi \rho_f c_t  \dfrac{p - p^t}{\tau} + \nabla \cdot \left ( -\dfrac{\rho_f k}{\mu} \nabla p^t \right ) = 0
	\end{multlined}
\end{equation*}

И интегрирование по частям:
\begin{equation*}
	\begin{multlined}
		\int_{\Omega} \left [ \phi \rho_f c_t  \dfrac{p - p^t}{\tau} q + \nabla \cdot \left ( -\dfrac{\rho_f k}{\mu} \nabla p^t \right ) q \right ] d \Omega = 	\\[10pt] = \phi \rho_f c_t \int_{\Omega} \dfrac{p q - p^t q}{\tau} d \Omega - \int_{\Omega } \left [ \left ( -\dfrac{\rho_f k}{\mu} \nabla p^t \right ) \cdot \nabla q \right ] d \Omega + \int_{\Omega } \left [ \left ( -\dfrac{\rho_f k}{\mu} \nabla p^t \cdot n\right ) q \right ] d \partial \Omega
	\end{multlined}
\end{equation*}


Система полученных уравнений была решена средствами Fenics и использовались слабые формы уравнений.

\subsection*{Результаты}

Предложенная задача была решена с параметрами представленными на \ref{tab:params}. Расчет был произведен для разных значений угла между линией перфорации и направлением, перпендикулярным направлению максимального напряжения. Для простоты реализации алгоритма, вместо вращения линии перфорации, вращалась ось координат так, чтобы имитировать различные углы перфорации. 

\begin{table}
	\centering
	\begin{tabular}{|| M{2.5cm} | M{2.5cm} | M{2.5cm} | M{2.5cm} | M{2.5cm} ||} 
		\hline
		Эксперимент № & $\theta$, [рад.] & $\sigma_{min}$ [Па] & $\sigma_{max}$ [Па] & $\epsilon_{max}$ \\ [2ex]
		\hline \hline
		1 & $\cfrac{\pi}{4}$ & $10^6$ & $5 \cdot 10^6$ & $ 95 \cdot 10^{-3}$ \\ [3ex]
		\hline
		2 & $\cfrac{\pi}{12}$ & $10^6$ & $5 \cdot 10^6$ & $ 95 \cdot 10^{-3}$\\ [3ex]
		\hline
		4 & $-\cfrac{\pi}{12}$ & $10^6$ & $5 \cdot 10^6$ & $ 95 \cdot 10^{-3}$\\ [3ex]
		\hline
		5 & $-\cfrac{\pi}{4}$ & $10^6$ & $5 \cdot 10^6$ & $ 95 \cdot 10^{-3}$\\ [3ex] 
		\hline
 	\end{tabular}
 	\caption{Параметры проведения экспериментов}
 	\label{tab:params}
\end{table}

Величины механических параметров:
\begin{equation*}
	E = 10 \cdot 10^{9} \text{ [Па] }, \nu = 0.25
\end{equation*}

Проницаемость и вязкость флюида:
\begin{equation*}
	k = 200 \text{ [мДа] }, \mu = 10^{-7} \text{ [м / с] [м] }
\end{equation*}

% \begin{figure}[h]
% 	\centering
% 	\includegraphics[width=0.65 \textwidth]{images/part2/crack_1.png}
% 	\caption{Эксперимент № 1, полученная трещина}
% 	\label{fig:crack_1}
% \end{figure}

% \begin{figure}[h]
% 	\centering
% 	\includegraphics[width=0.65 \textwidth]{images/part2/crack_2.png}
% 	\caption{Эксперимент № 2, полученная трещина}
% 	\label{fig:crack_2}
% \end{figure}

% \begin{figure}[h]
% 	\centering
% 	\includegraphics[width=0.65 \textwidth]{images/part2/crack_3.png}
% 	\caption{Эксперимент № 3, полученная трещина}
% 	\label{fig:crack_3}
% \end{figure}

% \begin{figure}[h]
% 	\centering
% 	\includegraphics[width=0.65 \textwidth]{images/part2/crack_4.png}
% 	\caption{Эксперимент № 4, полученная трещина}
% 	\label{fig:crack_4}
% \end{figure}

% \begin{figure}[h]
% 	\centering
% 	\includegraphics[width=0.65 \textwidth]{images/part2/crack_5.png}
% 	\caption{Эксперимент № 5, полученная трещина}
% 	\label{fig:crack_5}
% \end{figure}

\begin{figure}
	\begin{tabular}{cc}
		\includegraphics[width=0.45 \textwidth]{images/part2/crack_1.png} & \includegraphics[width=0.45 \textwidth]{images/part2/crack_2.png} \\
		Эксперимент № 1 & Эксперимент № 2 \\[6pt]
		\includegraphics[width=0.45 \textwidth]{images/part2/crack_3.png} &   \includegraphics[width=0.45 \textwidth]{images/part2/crack_4.png} \\
		Эксперимент № 3 & Эксперимент № 4 \\[6pt]
		\multicolumn{2}{c}{\includegraphics[width=0.45 \textwidth]{images/part2/crack_5.png} }\\
		\multicolumn{2}{c}{Эксперимент № 5}
	\end{tabular}
	\caption{Результаты: полученные трещины для различных наклонов линии перфорации}
\end{figure}

Как видно из 8 - направление трещины зависит от угла перфорации так, что сама трещина в процессе ГРП распространяется вдоль направления, перпендикулярного наибольшего напряжения - $\sigma_{max}$.

Для демонстрации итоговых результатов удобно отобразить промежуточные итоги на одном рисунке:
\begin{figure}
	\centering
	\includegraphics[width=0.65 \textwidth]{images/part2/cracks.png}
	\caption{Моделирование распространения трещин ГРП для различных значений угла перфорации}
	\label{fig:cracks}
\end{figure}



На \ref{fig:cracks} изображены трещины для разных значений угла перфорации, а также отображено направление, по которому предположительно трещина будет расти. Как можно видеть, данное предположение выполняется с использованием модели \eqref{eq:poroelasticity}.

\subsection*{Выводы}

В данном разделе была рассмотрена задача о распространении трещины ГРП выходящей из заранее сделанной перфорации, а также рассмотрена зависимость того, как трещина растет от величины угла перфорации. Для моделирования среды использоваласт простая линейная модель, рассмотрены хрупкие разрушения, а также в качестве модели фильтрации использовался закон Дарси. \newpage В дальнейшем, можно добавить зависимость проницаемости от степени разрушения элемента, были проведены попытки учесть данный эффект, однако на данный момент работа еще не доделана. Также стоит отметить тот факт, что трещина имеет сложную структуру - это связано с тем фактом, что использовался метод конечного элемента и элементы Лагранжа - кусочно-линейные функции, а также сетка, состоящая из треугольных элементов. Все эти факты приводят к некоторым вычислительным погрешностям. Также вдобавок к погрешностям численного решения, добавляется момент, связанный с визуализацией. 

% \chapter{Solving differential equations}

% Before starting talking about solving a single partial differential equation (PDE) or system of PDEs and about neural networks need to clearly understand what are the existing methods for this problem, how they work, what the strong and weak sides, which facts influence the quality of the solution. 

% \section{Function approximation}

% \subsection{Что-то про аппроксимацию и регуляризации в кратце}

Let's start from supervised learning and suppose the set of pairs is given:
\begin{equation*}
	D = \{ x^i, y^i \}_{i = 1}^N
\end{equation*}

where 
\begin{equation*}
y^i = f(x^i) + \epsilon
\end{equation*}

In other words, here is presented the dataset of function values in some nodes. In general case not important the dimension of $x$ and $y$, that is why the previous relation for $y$ can be rewritten as:
\begin{equation*}
	f: A \rightarrow B, \quad A \in R^n, B \in R^m
\end{equation*}

For simplicity, let $n = 1$, $m = 1$, for the other cases the same way.

There are a lot of ways to build the approximation, for example using linear model, Linear regression, or using more advanced techniques, Ridge regression or Neural Networks (NN). For example, Linear regression:
\begin{equation}
	\label{eq:linear_1d}
	\hat{y}^j = \beta_0 + \sum_{i = 1}^n \beta_i x^j_i = \beta_0 + \beta_1 x^j
\end{equation}

and the main goal is to estimate the coefficients $a_0$ and $a_1$. Here $n$ is the dimension of the $A$ space. If the dimension of $A$ is more than 1, the matrix form  is more suitable for \eqref{eq:linear_1d}:
\begin{equation}
	\label{eq:linear_matrix_form}
	\hat{y}^j = \beta_0 + \sum_{i = 1}^n \beta_i x^j_i = x^T \beta \implies Y = X \beta
\end{equation}


In the general case, \eqref{eq:linear_matrix_form} can be rewritten as:
\begin{equation}
	\label{eq:linear_expansion}
	\hat{y}^j = \beta_0 + \sum_{i = 1}^K \beta_i \phi_i(x^j_i) \quad \text{or} \quad Y = Z \beta, \text{where } Z^j_i = \phi_i(x^j_i)
\end{equation}

where the functions $\phi_j$ are predefined earlier depends on the specificity of the problem. $K$ is the count of the predefined functions.

For the regression problem and for coefficients estimation the quality function or loss function is needed to be defined. 
\begin{equation}
	R(x) = \hat{y} - y, \quad R(x^i) = R^i = \hat{y}^i - y^i
\end{equation}
residual at $x^i$. 
The main goal is to minimize the sum of residuals:
\begin{equation*}
	\sum_{x^i \in X} R(x) \rightarrow min, \quad \text{or }  \sum_{x^i \in X} g(R(x)) \rightarrow min
\end{equation*}
, $g$ is monotonic function - loss function.
The most widely used loss function for this type of problem is the mean squared error or $R^2$ score. In this work, the mean squared error will be used:

\begin{equation}
	\mathcal{L} = \dfrac{1}{N} \sqrt{\sum_{i = 1}^N \left ( y_i - \hat{y_i} \right )^2} = \dfrac{1}{N} \sqrt{\sum_{i = 1}^N \left ( R^i \right )^2}
	\label{eq:loss}
\end{equation}
where $y^i$ is the function value at $x^i$ from the dataset and $\hat{y^i}$ predicted from the model.

For the estimate, the coefficients use the least-squares method for \eqref{eq:linear_matrix_form}:
\begin{equation*}
	\dfrac{\partial \mathcal{L}}{\partial a_i} = \dfrac{\partial}{\partial a_i} \dfrac{1}{N} \sqrt{\sum_{i = 1}^N \left ( R(x^i) \right )^2}
\end{equation*}

The considered way very powerful for estimation coefficients, for analysis and can be effectively solved via linear algebra instruments. Using statistical methods the number of needed functions and their values estimates with their confidence intervals. The problem can arise, when the possibility to calculate the derivatives is absent or the extremum of the loss function is not unique.
% \subsection{Что то про линейную регрессию}
From \eqref{eq:linear_matrix_form} linear model is:
\begin{equation*}
	Y = X \beta
\end{equation*}
$\beta$ - unknown parameters. The residual for this model is:
Now, just substitute the residual to loss function \eqref{eq:loss}:
\begin{equation*}
	\mathcal{L} = \dfrac{1}{N} \sqrt{\sum_{i = 1}^N \left ( R^i \right )^2} \Leftrightarrow \| Y - X \beta \|^2 \rightarrow min
\end{equation*}
\begin{equation}
	\| Y - X \beta \|^2 = \left ( Y - X \beta \right )^T \left ( Y - X \beta \right ) =  Y^T Y - Y^T X \beta - \beta^T X^T Y + \beta^T X^T X \beta
	\label{eq:linear_opt}
\end{equation}
And compute $\dfrac{\partial \mathcal{L}}{\partial \beta}$:
\begin{equation*}
	\dfrac{\partial }{\partial \beta} \left [ Y^T Y - Y^T X \beta - \beta^T X^T Y + \beta^T X^T X \beta \right ] \implies X^T Y =  X^T X \beta \\
	\beta = \left ( X^T X \right )^{-1} X^T Y
\end{equation*}

The last operation was very dangerous in sense when the inverse matrix doesn't exist or ill-conditioned. For example, if the determinant of matrix $X^T X$ doesn't exist what should do? Or $X^T X$ is ill-conditioned?
The answer is to apply the special techniques to avoid it - regularization. 
Look at the \eqref{eq:linear_opt} and add the additional term \cite{kress2012numerical}:
\begin{equation}
	\| Y - X \beta \|^2 + \lambda \| \beta \|^2 \implies \beta = \left ( X^T X + \lambda I \right )^{-1} X^T Y
\end{equation}
From \cite{kress2012numerical} implies fact, that $X^T X + \lambda I$ is not singular matrix and in fact has better condition number than $X^T X$. Moreover, there are a lot of regularization techniques \cite{ridge}, \cite{lasso}, \cite{dantzig_selector}, \cite{rlad}, \cite{slope}. 
It is only one simple example of problems, arises during the approximation process. By the way, more powerful methods exist, avoiding some problems: Random Forest, Gradient Boosting\ cite{bishop}, Neural Networks \cite{haykin}.
\paragraph{Impact of the regularization}
Let $y = f(x) = k  x + \epsilon, k = 10$. At the fig. \ref{fig:regularizations} seen, that regularization impact is big. After applying linear models (\cite{ridge}, \cite{lasso}, \cite{rlad}) for this problem with different regularization methods was got a different results. The RLAD method estimate the $\hat{k} = 10.498$, Ridge method $\hat{k} = 9.109$ and Lasso - $\hat{k} = 9.604$. Results slightly different because the key difference is using different norms for regularization. It is an important fact for the next work, where more complex regression models will be used. 

\begin{figure}[h]
	\centering
	\includegraphics[width=\textwidth]{images/chapter2/reularization_impact.png}
	\caption{Comparison of different regularizations}
	\label{fig:regularizations}
\end{figure}

% \newpage
% \section{Expansion the functions into the functional series}
% Linear regression looks like the function expansion, in some sense, into the series of predefined functions \eqref{eq:linear_expansion}. If let $\phi_i = cos$ then the linear regression transforms to cosine expansion, but without orthogonality criterion. Instead of cosines can be used any functions, Chebyshev polynomials or Legendre polynomials. 
% Let's talk about each function separately.

% \subsection{Fourier series}
Instead of arbitrary $\phi_i$ substitute $e^{i \pi k}$ to \eqref{eq:linear_expansion}:
\begin{equation*}
	\begin{multlined}
		S_K(x) = \beta_0 + \sum_{i = 1}^K \beta_i e^{i \pi k x} \implies S(x) = a_0 + \sum_{i = 1}^K \left (a_i cos(\pi k x) + b_i sin(\pi k x) \right ) \\
	\end{multlined}
\end{equation*}
This set of functions is orthogonal on the considered interval:
\begin{equation*} 
	\int_{-\pi}^{\pi} e^{i \pi k x} e^{i \pi l x} = 2\pi \delta_k^l 
\end{equation*}
The estimation of $a_n$ and $b_n$ for some function $f$ is a straightforward process, first of all, define the residual for this expansion - $R(x) = f(x) - S_K(x)$, after that, integrate the squared residual over the domain and use the least-squares method:
\begin{equation*}
	\begin{multlined}
		\mathcal{L} = \int_{\Omega} R(x)^2 d\Omega = \int_{\Omega} \left ( f(x) - S_K(x) \right )^2 d\Omega = \\ = \int_{\Omega} f(x)^2 d\Omega - 2 \int_{\Omega} S_K(x) f(x) d\Omega + \int_{\Omega} S_K(x)^2 d\Omega = \\ = \int_{\Omega} f(x)^2 d\Omega - 2 \int_{\Omega} \left [ a_0 + \sum_{i = 1}^K \left (a_i cos(\pi k x) + b_i sin(\pi k x) \right ) \right ] f(x) d\Omega + \\ +  \int_{\Omega} \left [ a_0 + \sum_{i = 1}^K \left (a_i cos(\pi k x) + b_i sin(\pi k x) \right ) \right ]^2 d\Omega
	\end{multlined}
\end{equation*}
And the derivates of the integrated residual:
\begin{equation}
 	\begin{multlined}
 		\begin{cases}
 			\dfrac{\partial }{\partial a_n} \mathcal{L} = \dfrac{\partial \mathcal{L}}{\partial S_K(x)} \dfrac{\partial S_K(x)}{\partial a_n}  = -2 {\displaystyle \int_{\Omega}} f(x) cos(\pi k x) d\Omega + -2 {\displaystyle \int_{\Omega}} S_K(x) cos(\pi k x) d\Omega \\[20pt]
 			\dfrac{\partial }{\partial b_n} \mathcal{L} = \dfrac{\partial \mathcal{L}}{\partial S_K(x)} \dfrac{\partial S_K(x)}{\partial b_n} = -2 {\displaystyle \int_{\Omega}} f(x) sin(\pi k x) d\Omega + -2 {\displaystyle \int_{\Omega}} S_K(x) sin(\pi k x) d\Omega \\[20pt]
 			\dfrac{\partial }{\partial a_0} \mathcal{L} = \dfrac{\partial \mathcal{L}}{\partial S_K(x)} \dfrac{\partial S_K(x)}{\partial a_0}  = -2 {\displaystyle \int_{\Omega}} f(x) d\Omega + 2 |\Omega| a_0 \\[20pt]
 		\end{cases}
 	\end{multlined}
 	% \mathcal{L} = \int_{\Omega} R(x)^2 d\Omega
\end{equation}
The main part of the calculations is absent and provided here \cite{fourierintro}. In addition, there is a convergence analysis of the coefficients, in sense of pointwise, and $L_2$ norm. The final result for the coefficients is:
\begin{equation}
	\begin{cases}
		a_0 = \dfrac{{\displaystyle \int_{\Omega}} f(x) d\Omega }{2 | \Omega |} \\[20pt]
		a_n = \dfrac{{\displaystyle \int_{\Omega}} f(x) cos(\pi k x) d\Omega}{| \Omega |} \\[20pt]
		b_n = \dfrac{{\displaystyle \int_{\Omega}} f(x) sin(\pi k x) d\Omega}{| \Omega |} \\[20pt]
	\end{cases}
\end{equation}

\paragraph{Example of function expansion into the Fourier series} 
Consider the function $y = sin(x) + x$ and at the fig. \ref{fig:fourier_demo} the results. It can be seen that with a relatively small amount of the terms the approximation good. With 3 terms the approximation error\footnote{Here, the approximation error is the loss function and concrete - mean squared error between the known values and Fourier expansion. There is a pointwise loss value, where the error calculation includes the finite number of nodes and integral loss value, where the residual integrates over the all domain.} is 0.565, 5 terms - 0.005 and 10 terms is 0.002.
\begin{figure}[h]
	\centering
	\includegraphics[width=0.65 \textwidth]{images/chapter2/fourier_demo.png}
	\caption{Example of function expansion into the Fourier series with 3, 10, 18 terms}
	\label{fig:fourier_demo}
\end{figure}

\subsubsection{The strong sides of Fourier expansion}
There are two important theorems helps to use Fourier expansion for construction of the future approximation:

\newtheorem{theorem}{Theorem}[chapter]
\begin{theorem}
\label{convergence-l2-norm}
If $f$ belongs to $L^{2}(\left[-\pi ,\pi \right])$ then $S_k$ converges to $f$ in $L^{2}(\left[-\pi ,\pi \right])$, that is, $\|S_K - f\|_{2}$ converges to 0 as $N \rightarrow \infty$.
\end{theorem}
\begin{theorem}
\label{convergence-pointwise}
If $f$ belongs to $C^1(\left[-\pi ,\pi \right])$ then $S_k$ converges to $f$ uniformly (and hence also pointwise).
\end{theorem}
The proofs of theorems well provided here \cite{fourierintro}. And an additional fact, Fourier coefficients of any integrable function tend to zero. 


\begin{figure}[h]
	\centering
	\includegraphics[width=0.65 \textwidth]{images/chapter2/fourier_quality.png}
	\caption{Illustration of the theorems \ref{convergence-l2-norm}, \ref{convergence-pointwise} for the function from the previous example}
	\label{fig:fourier_quality}
\end{figure}
% \subsection{Chebyshev polynomials and series}
Again, instead of using trigonometric polynomials, try to apply another system of orthogonal polynomials - Chebyshev polynomials and expansion into the Chebyshev series. 
The Chebyshev polynomials can be defined as the recurrent sequence:
\begin{equation}
	T_0(x) = 1, T_1(x) = x, \dots, T_n(x) = 2 x T_{n - 1}(x) - T_{n - 2}(x)
\end{equation}
This polynomials are orthogonal with weight $w(x) = \dfrac{1}{\sqrt{1 - x^2}}$\cite{mason2002chebyshev}:
\begin{equation}
	\label{eq:chebyshev_series}
	\int_{-1}^{1} T_k(x) T_l(x) w(x) dx = \begin{cases}
		\pi \delta_l^k, k = 0, \\
		\cfrac{1}{2} \pi \delta_l^k, k \neq 0
	\end{cases}
\end{equation}
Examples of the first six polynomials are plotted at the \ref{fig:chebyshev_demo}
\begin{figure}[h]
	\centering
	\includegraphics[width=0.65 \textwidth]{images/chapter2/chebyshev_demo.png}
	\caption{Chebyshev polynomias for n = 3, 4, 5, 6}
	\label{fig:chebyshev_demo}
\end{figure}

These polynomials are used for the interpolation procedure to avoid the Runge phenomenon\footnote{In the mathematical field of numerical analysis, Runge's phenomenon is a problem of oscillation at the edges of an interval that occurs when using polynomial interpolation with polynomials of high degree over a set of equispaced interpolation points.}, in case, when they are monic polynomials. Moreover, the roots of them often used in numerical linear algebra in method conjugated gradient descent, for example\footnote{
More information is here - ``E. Kaporin, Using Chebyshev polynomials and approximate inverse triangular
factorizations for preconditioning the conjugate gradient method, Zh. Vychisl. Mat.
Mat. Fiz., 2012, Volume 52, Number 2, 179–204``}.
The details of the Chebyshev polynomials is not important for this work, but there are a lot of benefit of using them in different problems.

Chebyshev series is the expansion of the function by his polynomials or, simply substitute polynomials to \eqref{eq:linear_expansion}:
\begin{equation*}
	\begin{multlined}
		f(x) = \sum_{k = 0}^{\infty} c_k T_k(x) \\
		c_k = \dfrac{1}{M} \int_{-1}^1 f(x) T_k(x) w(x) dx, \text{where } M = \begin{cases}
		\pi, k = 0, \\
		\dfrac{1}{2} \pi, k \neq 0
	\end{cases} \\
	\text{and } w(x) \text{ weight function } = \dfrac{1}{\sqrt{1 - x^2}}
	\end{multlined}
\end{equation*}

And the convergence theorem.
\begin{theorem}
\label{chevyshev_convergence}
When a function $f$ has $m + 1$ continous derivatives on $[-1, 1]$ or $ f \in C^{m + 1}[-1, 1] $, where $m \in N^+$, then $\| f(x) - S_k(x) \| = \mathcal{O}\left ( \dfrac{1}{k^m} \right )$ as $k \rightarrow \infty \quad \forall x \in [-1, 1]$ 
\end{theorem}
The proof here \cite{mason2002chebyshev}.
This theorem described the same fact that theorem \ref{convergence-l2-norm}  described for the Fourier expansion.

\paragraph{Example of Chebyshev interpolation}
Consider the function $y = sin(x) + x cos(x)$ and at the fig. \ref{fig:chebyshev_expansion} the results. 
\begin{figure}[h]
	\centering
	\includegraphics[width=0.7 \textwidth]{images/chapter2/chebyshev_series.png}
	\caption{Chebyshev expansion with 3, 4, 5, 6 terms }
	\label{fig:chebyshev_expansion}
\end{figure}


In fact, Chebyshev series is Generalized Fourier series: 
\begin{equation*}
	\begin{multlined}
		f(x) = \sum_{i = 1}^{N} c_n \phi_n(x) \\
		\left < \phi_i, \phi_j \right > = \int_{V} \phi_i \phi_j w dV = K \delta_i^j
	\end{multlined}
\end{equation*}

% \subsection{Sigmoidal Functional Series}
Let $\sigma = \dfrac{1}{1 + e^{-x}}$ and substitute to \eqref{eq:linear_expansion} instead of $\phi_i$ and apply an affine transformation to argument:
\begin{equation}
	\label{eq:sigmoidal_expansion}
	G(x) = \sum_{i = 1}^K \alpha_i \sigma(\beta_i x^j_i + \gamma_i)
\end{equation}
the expansion over the sigmoid functions is got. 
This expansion, in fact, one of the widely used in approximation process. First of all, there is a useful theorem that provides guarantees of the quality for approximation. 

\begin{theorem}
	\label{sigmoidal_expansion}
	Let $\sigma = \dfrac{1}{1 + e^{-x}}$, then finite sums of the form:
	\begin{equation*}
		G(x) = \sum_{i = 1}^K \alpha_i \sigma(\beta_i x^j_i + \gamma_i)
	\end{equation*}
	are dense in $C(I_n\footnote{
	 Unit cube in $R^n$. The term unit cube or unit hypercube is also used for hypercubes, or "cubes" in $n$-dimensional spaces, for values of n other than 3 and edge length 1
	})$. In other words, given any $f \in C(I_n)\footnote{
		The space of continuous functions over $I_n$
	}, \epsilon > 0$, there is a sum, $G(x)$, of the above form, for which:
	 $\| G(x) - f(x) \| < \epsilon, \quad \forall x \in I_n$
\end{theorem}
Simply put, this theorem provides the rationale for expanding a function in a sigmoid series, in addition, the quality of approximation can be significantly improved by increasing the terms in the series. By the way, the \eqref{eq:sigmoidal_expansion} series is also called the single-layered perceptron \footnote{In machine learning, the perceptron is an algorithm for the controlled training of binary classifiers or regressors.} Or the artificial neural network \footnote{Neural network, or Artificial neural networks (ANN ), or connection systems, are computing systems vaguely inspired by the biological neural networks that make up the brain of animals. Such systems "learn" to perform tasks, examining examples, usually without programming, using rules specific to the tasks.}

Coefficients in the expansion can also be found using the least squares method or using gradient-based methods that are better suited when it comes to artificial neural networks. The question of estimating the coefficients (or weights) is a question of the future section, but there are many special algorithms for this.

\subsection{Another functions and expansion over them}
In case, when Generalized Fourier Series (GRS) is considered there are a lot of function can be used, for example \cite{abramowitz1965handbook}:
\begin{itemize}
	\item Gegenbauer polynomials
	\item Jacobi polynomials
	\item Romanovski polynomials
	\item Legendre polynomials
\end{itemize}

All of them have their own weight functions and are generalized Fourier series. Potentially, each of these expansions is applicable for approximating functions using the least squares method and using methods based on gradient descent with an appropriate step and regularization methods. In fact, in the process of approximation (or controlled learning), the basis function is fixed and the appropriate regularization method is selected, for example, the Lagrange method \cite{городецкий2007нелинейное}. Moreover, the use of orthogonal functions for approximation or interpolation is not limited to generalized Fourier series. In practice, linearly independent functions or sets of functions are most often used based on prior knowledge of the approximated function or process.

% \newpage
% \section{Solving differentials equations}
% \subsection{Introduction}
Differential equations can be split into two big groups:
\begin{itemize}
	\item Ordinary differential equations (ODE)
	\begin{itemize}
		\setlength\itemsep{0.5em}
		\item Single ODE
		\item System of ODEs
	\end{itemize}
	\item Partial differential equations (PDE)
	\begin{itemize}
		\setlength\itemsep{0.5em}
		\item Single PDE
		\item System of PDEs
	\end{itemize}
\end{itemize}
For each group or class of equations, there are many solution methods: for ordinary differential equations, use the shooting method, Euler's method, Runge-Kutta methods, etc., for differential equations in partial derivatives, methods of finite differences, finite elements, finite volumes, etc. are used. Most of them are based on the idea of ​​integrating or approximating functions numerically through a sequence of functions and minimizing the residuals, for example, a group of methods called weighted residual methods \cite{finlayson2013method} \cite{fletcher2012computational}. At the heart of all methods, the key step is to solve a system of algebraic equations in the general case of nonlinear ones. In the case when the equation is quite simple, the system of linear equations must be solved quickly and efficiently, but poor conditionality is one example of what would be required using special pre-conditioning approaches. Suppose that after applying some numerical method for some problem and it doesn’t matter for which method and which task, we get a linear system: $A x = b$, let $b = \hat{b} + e_b $, where $e_b$ is error in vector $b$ it can be caused by rounding errors or predefined errors related to data collection if speech going about real problems of the oil and gas industry, for example. Here the existence of this error is interesting because the solution error implies from the error in the left-hand side and can be larger than her. $x = A^{-1} (\hat{b} + e) = A^{-1} \hat{b} + A^{-1} e_b$, and $x = \hat{x} + e_x = A^{-1} \hat{b} + A^{-1} e_b$ and:
\begin{equation*}
	\begin{cases}
		\hat{x} = A^{-1} \hat{b} \\
		e_x = A^{-1} e_b
	\end{cases} \implies \max_{e, b} \dfrac{\|A^{-1} e_b\|}{\|A^{-1} \hat{b} \|} \dfrac{\|b\|}{\|e_b\|} = \| A \| \| A^{-1} \| = \kappa(A)
\end{equation*}
$\kappa(A)$ - condition number\cite{gentle2007matrix}. 

It means that if the matrix has a large value of the condition number then the error of the $x$ is large\footnote{The influence of the condition number on the solution accuracy presented at the fig. \ref{fig:ill_condition_demo}. Here considered the model example of the linear system, with $\kappa = 3422.83$ and presented the components of the vector b and his deviations, then x was found and deviations. It can be seen that the deviations of b (left part, blue circles) near the initial values(red line), but the deviations for x has a large spreading. This is an influence of condition number.}.
This fact leads to the use of preconditioners to decrease the condition number and gets a more stable solution. There are a lot of ways to preconditioning the system of linear equations: Jacobi (or diagonal) preconditioner, incomplete Cholesky factorization, incomplete LU factorization, and so on. 

This is one of the problems that arise when solving ordinary differential equations or partial differential equations, but in fact there are problems such as convergence rate, grid generation, solution interpolation, choice of function for approximation, and so on. This work does not propose a method that is ideal and works well for all tasks, but for the tasks presented below, the method using artificial neural networks works sufficiently accurately and quickly, completely eliminating the need to solve a system of algebraic equations. In fact, if the method does not depend on the grid and uses only randomly selected points and works well for some tasks, this already means that in this direction it is possible to develop and prove the quality of work / convergence / uniqueness.

\begin{figure}[h]
	\centering
	\includegraphics[width=\textwidth]{images/chapter2/ill_condition_demo.png}
	\caption{The influence of the condition number on the solution accuracy}
	\label{fig:ill_condition_demo}
\end{figure}
% \subsection{Solving Ordinary differential equations}
In this part consider the arbitrary ODE:
\begin{equation}
	\label{eq:ode}
	\begin{multlined}
		\mathcal{D}\left (x, y, y^{(1)}, \dots, y^{(n)}\right) = 0, \quad x \in \Omega = [0, 1] \subset  R \\
		B(y) = 0, \quad x \in \partial \Omega = \{0, 1\}
	\end{multlined}
\end{equation}
, where $\mathcal{D}(\dots)$ - differential operator, $B(y)$ - boundary conditions function. 
\begin{equation*}
	B(y) = \begin{cases}
		D(y) = 0, \quad x \in \partial \Omega_D = \{0, 1\} \\
		N(y) = 0, \quad x \in \partial \Omega_N = \{0, 1\} 
	\end{cases}, \partial \Omega_N \cup \partial \Omega_D = \partial \Omega
\end{equation*}
where $D(y)$ - Dirichlet boundary conditions and $\partial \Omega_D$ boundary for these conditions, N(y) - Neumann boundary conditions and $\partial \Omega_N$ is bound them.
All of the equations considered now and in the future in this work will be defined as \eqref{eq:ode}.

For solving this equation will be considered two methods, weighted residuals method \cite{fletcher2012computational} (Bubnov-Galerkin) and finite differences method \cite{dimov2019finite}.

For concreteness, ODE example for consideration:
\begin{equation*}
	\begin{multlined}
		\dfrac{d}{d x^2} y(x) + 2y = sin(2x) \left ( 1 - 2 sin(2x) \right), x \in [0, 1]\\
		y(0) = 0, \quad \dfrac{d}{dx} y \Big|_{x = 0} = 1
	\end{multlined}
\end{equation*}
, where $y = \dfrac{1}{2} sin(2x)$ - analytical solution.

\subsubsection{Galerkin method}
The key idea is to define the solution as:
\begin{equation}
	\label{eq:galerkin_presentation}
	y_h = \phi_0 + \sum_{i = 1}^N a_i \phi_i(x)
\end{equation}
and $\phi_0$ satisfy all boundary conditions $\phi_0: B(\phi_0) = 0 \implies \phi_0(0) = 0, \phi_0(1) = 1$ and $\phi_i$ satisfy the homogenous boundary conditions $\phi_i(0) = \phi_i(1) = 0$. The solution to the problem is a some weighted sum of linearly independent functions that satisfy the boundary conditions and in the general case satisfy the initial conditions too. For calculation, the coefficients use minimization residual method, where:
\begin{equation*}
	\min_{a_1, \dots, a_n} R(a_1, \dots, a_n) = \int_{\Omega} \mathcal{D}(y_h) d\Omega + \int_{\partial \Omega_N} N(y_h) d\partial \Omega
\end{equation*}
From \cite{fletcher2012computational} is known that $R$ does not equal zero in the general case and for evaluating the coefficients use the integration with weight function:
\begin{equation*}
	\int_{\Omega} w R(a_1, \dots, a_n) d\Omega = \int_{\Omega} w \mathcal{D}(y_h) d\Omega + \int_{\partial \Omega_N} w N(y_h) d\partial \Omega = 0
\end{equation*}
, where $w = \psi_i$, weight functions. The residual and weight functions must be orthogonal.
In a more general case without using $phi_0$ the residual is:
\begin{equation*} 
	\begin{multlined}
		\int_{\Omega} \psi_j R(a_1, \dots, a_n) d\Omega = \int_{\Omega} \psi_i \mathcal{D} \left ( \sum_{i = 1}^N a_i \phi_i(x) \right ) d\Omega + \int_{\partial \Omega} \psi_j B \left ( \sum_{i = 1}^N a_i \phi_i(x) \right ) d\partial \Omega = \\ =
		\int_{\Omega} \psi_j \mathcal{D} \left ( \sum_{i = 1}^N a_i \phi_i(x) \right ) d\Omega + \int_{\partial \Omega} \psi_j B \left ( \sum_{i = 1}^N a_i \phi_i(x) \right ) d\partial \Omega = \text{ if $\mathcal{D}, \mathcal{B}$ are linear operators } = \\ = \int_{\Omega} \sum_{i = 1}^N  a_i \psi_j \mathcal{D}(\phi_i(x)) d\Omega + \int_{\partial \Omega} \sum_{i = 1}^N  a_i \psi_j B(\phi_i(x)) d\partial \Omega = 0
	\end{multlined}
\end{equation*}
From the equation above system of algebraic equations can be constructed and solve it for unknown coefficients $a_i$.
\subsubsection{Galerkin method, special case}
If $w$ is delta Dirac function, then the Galerkin method also called the Pointwise collocation method, which more easy for implementation.
\begin{equation}
	\begin{multlined}
		w = \delta(x - x_k), x_k \in X \subset \Omega, \text{ and } \| X \| = K, \\
		\int_{\Omega} \delta(x - x_k) R(a_1, \dots, a_n) d\Omega = \int_{\Omega} \delta(x - x_k) \mathcal{D} \left ( \sum_{i = 1}^N a_i \phi_i(x) \right ) d\Omega + \\ + \int_{\partial \Omega} \delta(x - x_k) B \left ( \sum_{i = 1}^N a_i \phi_i(x) \right ) d\partial \Omega
		 = \mathcal{D} \left ( \sum_{i = 1}^N a_i \phi_i(x_k) \right ) + B \left (\sum_{i = 1}^N a_i \phi_i(x_k) \right )
	\end{multlined}
\end{equation}
In case when the number of points of collocation more then the number of unknown coefficients the problem solves via optimization techniques, the least-squares method for example.

\subsubsection{Finite difference method}
First of all, the finite difference derivative is:
\begin{itemize}
	\item Left derivative \begin{equation}
		\label{eq:left_der}
		\dfrac{d y}{d x} \Big|_{x = x_i} = \dfrac{y(x_i) - y(x_{i - 1})}{x_i - x_{i - 1}}
	\end{equation}
	\item Central derivative \begin{equation}
		\label{eq:central_der}
		\dfrac{d y}{d x} \Big|_{x = x_i} = \dfrac{y(x_{i + 1}) - y(x_{i - 1})}{x_{i + 1} - x_{i - 1}}
	\end{equation}
	\item Right derivative \begin{equation}
		\label{eq:right_der}
		\dfrac{d y}{d x} \Big|_{x = x_i} = \dfrac{y(x_{i + 1}) - y(x_i)}{x_{i + 1} - x_i}
	\end{equation}
\end{itemize}
Actually, the approximation quality better for the central difference derivative. The derivatives of higher order can be constructed from the first-order derivatives (left, right, central). For example:
\begin{equation*}
	\dfrac{d^2 y}{d x^2} \Big|_{x = x_i} = \dfrac{d}{d x} \Big|_{x = x_i} \left [ \dfrac{y(x_{i + 1})}{x_{i + 1} - x_i} \right ] - \dfrac{d}{d x} \Big|_{x = x_i} \left [ \dfrac{y(x_{i - 1})}{x_i - x_{i - 1}} \right ] = \dfrac{y(x_{i + 1}) - y(x_i)}{x_{i + 1} - x_i} - \dfrac{y(x_i) - y(x_{i - 1})}{x_i - x_{i - 1}}
\end{equation*}
When the grid uniformly distributes the $x_i$ values: $x_{i + 1} - x_i = d$:
\begin{equation*}
	\dfrac{d^2 y}{d x^2} \Big|_{x = x_i} = \dfrac{y(x_{i + 1}) - 2 y(x_i) + y(x_{i - 1})}{d^2}
\end{equation*}
So, the idea of the FDM is to substitute the finite derivatives and solve algebraic equations. 
\begin{equation*}
	\mathcal{D} (x, y, y^{(1)}, \dots, y^{(n)}) \Big|_{x = x_i} = \mathcal{D} \left ( x_i, y(x_i), \dfrac{y(x_{i + 1}) - y(x_{i - 1})}{x_{i + 1} - x_{i - 1}}, \dots, \dfrac{y(x_{i + n - 1}) + \dots + y(x_{i - n + 1})}{d^n} \right )
\end{equation*}
And the same way for the boundary conditions:
\begin{equation*}
	B(y) \Big|_{x = x_i}  = \begin{cases}
		D(y) = 0, \quad x \in \partial \Omega_D = \{0, 1\} \\
		N(y) = 0, \quad x \in \partial \Omega_N = \{0, 1\} 
	\end{cases} 
\end{equation*}
After the solving equations, the values of $y_i$ are known and needed to be interpolated over the domain $\Omega$.
\subsubsection{Comparison of the provided methods}	
Methods are very different, the FDM provides the solution in the fixed nodes and interpolates the solution from these nodes overall domain, on the other hand, the Galerkin method provides approximation solution in the mean sense over the domain. This difference makes the variability of the interpolation methods or basis functions for calibration of the numerical solution quality. The strong and ill sides of the FDM are high quality of the solution over the nodes, but the interpolation process leads to the Runge phenomenon, besides the size of the grid has a tremendous influence on the solution quality. 
The Galerkin method provides the approximation over the domain and strongly depends on the initial choice basis functions, so, there is the probability, that solution has a compact form.

It will be good if the strong sides of these methods will be combined into one approximator. Ideal case, when the number of terms increases, the solution quality increase too.

First of all, using the theorem \ref{sigmoidal_expansion} and the solution form \eqref{eq:galerkin_presentation}:
\begin{equation}
	\label{eq:perceptron_ode}
	y_h(x) = \phi_0(x) + \sum_{i = 1}^K \alpha_i \sigma(\beta_i x + \gamma_i)
\end{equation}
$\phi_0$ also satisfy the boundary conditions. 
For this solution from the theorem known, that the approximation quality strongly depends on the number of terms in the series, in addition, this form satisfies the boundary conditions, as in the Galerkin method. Now, the quality of the solution is guaranteed by the theorem and the question about basis function is solved. Moreover, using points collocation method:
\begin{equation}
	\label{eq:loss_galrekin}
	\begin{multlined}
		\mathcal{L} = \dfrac{1}{| X |} \sum_{x \in X} \left [ \| R(x; p_1, 
		\dots, p_N) \|^2 \right ], \quad X \in \Omega \subset R, p_i = (\alpha_i, \beta_i, \gamma_i) \in P \subset R^3 \\
		\textbf{Coefficients :} \min_{p_i} \mathcal{L} = \begin{cases}
			\dfrac{\partial \mathcal{L}}{\partial \alpha_i} = 0 \\[10pt]
			\dfrac{\partial \mathcal{L}}{\partial \beta_i} = 0 \\[10pt]
			\dfrac{\partial \mathcal{L}}{\partial \gamma_i} = 0
		\end{cases}
	\end{multlined}
\end{equation}
Currently, the solution is found using the least-squares method, which leads to solving the system of equations with not one solution. For each solution, the loss function should be calculated and chose the parameter where problem has a minimum value.

For this approach calculation of the derivatives for a differential operator should be provided:
\begin{equation}
	\label{eq:bad_system}
	\begin{multlined}
		\dfrac{d y_h}{d x} = \dfrac{d}{d x} \left [ \phi_0(x) + \sum_{i = 1}^K \alpha_i \sigma(\beta_i x + \gamma_i) \right ] = \dfrac{d}{d x} \phi_0(x) + \sum_{i = 1}^K \alpha_i \dfrac{d}{d x} \sigma(\beta_i x + \gamma_i) = \\ = \dfrac{d}{d x} \phi_0(x) + \sum_{i = 1}^K \alpha_i \beta_i  \sigma(\beta_i x + \gamma_i) (1 - \sigma(\beta_i x + \gamma_i))
	\end{multlined}
\end{equation}
The form of the derivative immediately told that the solving of equations \eqref{eq:bad_system} is very unstable and there are a lot of roots. On the other hand, using the numerical derivative \eqref{eq:left_der}, \eqref{eq:central_der}, \eqref{eq:right_der} leads to:
\begin{equation}
	\label{eq:good_system}
	\begin{multlined}
		\dfrac{d y_h}{d x} \Big|_{x = x_i} \approx \dfrac{y_h(x_{i + 1}) - y_h(x_{i - 1})}{2d} = \dfrac{1}{2d} \left [ y_h(x_{i + 1}) - y_h(x_{i - 1}) \right ]
	\end{multlined}
\end{equation}

\subsubsection{Artificial neural networks (ANN)}	
Definition from Wikipedia is ``Artificial neural networks (ANN) or connectionist systems are computing systems vaguely inspired by the biological neural networks that constitute animal brains. Such systems "learn" to perform tasks by considering examples, generally without being programmed with task-specific rules``, or the second one definition: ``A mathematical model, as well as its software or hardware implementation, built on the principle of organization and functioning of biological neural networks - networks of nerve cells of a living organism.''
\begin{figure}[h]
	\centering
	\includegraphics[width=1 \textwidth]{images/chapter2/simple_net.png}
	\caption{The illustration of \eqref{eq:perceptron_ode}. One layered neural network}
	\label{fig:simple_net}
\end{figure}


These definitions are similar, in the sense that the input signal passes through the set of ordered simple operations or layers, and at the end of these operations the output is the result of the neural network. The order of these operations also called the architecture of the neural network. There are a lot of different types of layers\footnote{The zoo of neural network types: \href{https://www.asimovinstitute.org/neural-network-zoo/}{ANN zoo}}, the most widely used is the fully connected layer or dense layer as in figure \ref{fig:simple_net}.
Looking more precisely the neural network is sequence of affine transformations (edges) and nonlinear transformation (nodes):
\begin{equation*}
	\begin{multlined}
		\mathcal{N}(x) = \left [ A^2 \circ \phi^1 \circ A^1 \right ] (x) = A^2 \phi^1 \left (A^1 x + b^1 \right ) + b^2 \\ A^1 \in R^{m \times n}, A^2 \in R^{k \times m}, b^1 \in R^m, b^2 \in R^k, x \in R^n
	\end{multlined}
\end{equation*}
In general case $l$ layered neural network is:
\begin{equation}
	\label{eq:neural_net}
	\mathcal{N} = A^l \circ \phi^{l - 1} \circ A^{l - 1} \circ \dots \circ \phi^1 \circ A^1 = A^l \left [ \phi^{l - 1} \left [ \dots \left [ A^1 (x) + b^1 \right ] \dots \right ] + b^{l - 1} \right] + b^l
\end{equation}
where $A^i, \forall i \in \{1, \dots, l\}$ is the parameter that must be found. For the successful using the neural networks:
\begin{itemize}
	\item Define the architecture
	\item Define the loss function
	\item Choose a suitable optimization algorithm \begin{itemize}
		\item How the optimization process looks
		\item Existing optimization algorithms
	\end{itemize}
\end{itemize}

\subsubsection{Optimization part. Backpropagation algorithm}

The main goal is getting the numerical solution of the DE and for this aim is to use the residual \eqref{eq:loss_galrekin} and minimize it over the parameters of the neural network:
\begin{equation*}
	\min_{A^l, b^l, \dots, A^1, b^1} \mathcal{L} = \min_{A^l, b^l, \dots, A^1, b^1} \mathcal{L} = \dfrac{1}{| X |} \sum_{x \in X} \| R(x) \|^2
\end{equation*}
Now, how to minimize this complex function? Using the least-squares leads to solving the equations or use gradient-based optimization. For the estimation, the values of the neural network parameters use the gradient-based methods and iteratively goes to the local minimum(!). Suppose, the for the point collocation method randomly choose the set of points at the $k$-th step, the loss is calculated and gradients are calculated:.
\begin{equation}
	\nabla A_k^l = \nabla_{A^l} \mathcal{L}_k, \quad A_{k + 1}^l = A_k^l - \lambda(k) \psi(\nabla A_k^l)
\end{equation}
the $\psi$ is the main part of the particular algorithm because using the $\psi(x) = x$, stochastic gradient descent (SGD) immediately have gotten. Using different $\psi$, the corresponding methods are obtained \cite{Adadelta}, \cite{Adagrad}, \cite{Adam}, \cite{Diffgrad}.
\paragraph{Optimizers comparison}
To demonstrate the quality of various optimization algorithms, a simple neural network architecture was chosen and trained to approximate the function. Lines are the average value of the loss function at a particular iteration, the region of the corresponding color is the region in which the error may lie on average. To collect such statistics, the neural network was trained by each optimizer 25 times.
The results are presented in the figure \ref{fig:optimizers}.
\begin{figure}[h]
	\centering
	\includegraphics[width=0.75 \textwidth]{images/chapter2/optimizers.png}
	\caption{Comparison of different optimizers for fixed neural network architecture}
	\label{fig:optimizers}
\end{figure}

Consider the sequence of the operators \eqref{eq:neural_net} and the quality function or loss function $\mathcal{L}$. Currently not important what loss function and the nature of the operators:
\begin{equation*}
	\begin{multlined}
		\mathcal{N} = A^l \circ \phi^{l - 1} \circ A^{l - 1} \circ \dots \circ \phi^1 \circ A^1, \quad \mathcal{L} = \mathcal{L} \left ( \mathcal{N} \right )
	\end{multlined}
\end{equation*}
For efficient evaluating the gradients over the parameters exists a backpropagation\footnote{
	In machine learning, backpropagation (backprop, BP) is a widely used algorithm in training feedforward neural networks for supervised learning. Generalizations of backpropagation exist for other artificial neural networks (ANNs), and for functions generally – a class of algorithms referred to generically as "backpropagation" - from Wikipedia
} algorithm \cite{chauvin2013backpropagation}. The key idea is to use the chain rule for the derivative:
\begin{equation*}
	\begin{cases}
		\dfrac{\partial \mathcal{L}}{\partial A^l} = \nabla_{A^l}\mathcal{L} \\[10pt]
		\dfrac{\partial \mathcal{L}}{\partial A^{l - 1}} = \left [ \phi^l \right ]^{'} \left [ A^{l - 1} \right ]^T \nabla_{A^l}\mathcal{L} \\[10pt]
		\dfrac{\partial \mathcal{L}}{\partial A^{l - 2}} = \left [ \phi^{l - 1} \right ]^{'} \left [ A^{l - 2} \right ]^T  \left [ \phi^l \right ]^{'} \left [ A^{l - 1} \right ]^T \nabla_{A^l}\mathcal{L} =  \left [ \phi^{l - 1} \right ]^{'} \left [ A^{l - 2} \right ]^T  \dfrac{\partial \mathcal{L}}{\partial A^{l - 1}} \\[10pt] 
		\text{For k-th derivative in the same way:} \\[10pt]
		\dfrac{\partial \mathcal{L}}{\partial A^{l - k}} = \left [ \phi^{l - k + 1} \right ]^{'} \left [ A^{l - k} \right ]^T  \dfrac{\partial \mathcal{L}}{\partial A^{l - k + 1}}
	\end{cases} 
\end{equation*}
Now it is known how a neural network works, how it is trained and why a solution can be built with arbitrary accuracy. Next, we will consider different architectures of neural networks for solving different problems, and different approaches, for example, the approach based on the Galerkin method, when the Dirichlet boundary conditions are embedded in a neural network. There is also an approach based on the Ritz method that reduces the solution of the equation to an extremal problem. For example, when solving equations, it is possible to integrate the boundary conditions into the approximator structure \cite{Lagaris_1998} \cite{liu2019solving}.
Here the solution is presented in the form:
\begin{equation}
	\label{eq:simple_solver}
	y_h = A(x) + B(x) \mathcal{N}(x)
\end{equation}
where $A$ satisfies the boundary conditions of the first and second kind, where $A$ satisfies the boundary conditions of the first and second kind, and $B$ is in a sense a function of distance, or rather a function that “removes” the values of the model (neural network) at the boundary. 
\paragraph{Example}
Consider equation $\phi \left ( x, y, \dfrac{d y}{d x}, \dfrac{d^2 y}{d x^2} \right ) = 0$ and boundary conditions $y(0) = y_0, y(1) = y_1$. In this case, the solution will be built in the form:
\begin{equation*}
	y_h = (1 - x) y_0 + x y_1 + (1 - x) x \mathcal{N}(x)
\end{equation*}
Thus, for such a form, a neural network is only part of the solution, for points within a region. It is clear that the name of the complex boundary conditions for the partial differential equation to construct a solution in this form is very difficult. In this form, it is convenient to search for a solution having homogeneous boundary conditions of the first kind. You can use the results from \cite{fletcher2012computational}, where it is proposed to construct the solution in such a way as to satisfy only conditions of the first kind, and transfer conditions of the second and third kind to the neural network again (to the loss function), example:
\begin{equation*}
	\begin{multlined}
		\phi \left ( x, y, \dfrac{d y}{d x}, \dfrac{d^2 y}{d x^2} \right ) = 0, \quad y(0) =  y_0, \dfrac{d y}{d x} \Big|_{x = 0} = y_1 \\
		y_h = (1 - x) y_0 + B(x) \mathcal{N}(x), \quad \mathcal{L}^{'} = \mathcal{L} + \lambda \left \| \dfrac{d y_h}{d x}\Big|_{x = 0} - y_1  \right \|
	\end{multlined}
\end{equation*}
Another approach \cite{cao2016locally} also embeds the boundary conditions in the general solution, however, it occurs due to an additional term that estimates the error between the conditions and the solution itself at the boundary and embeds the additional term in a row in order to satisfy the conditions. In fact, every few iterations of the network training, the term is recalculated (a small system of equations is solved) and adjusted to the boundary conditions. Not quite an easy way to implement, however, the quality of the final solution depends on the boundary conditions, on the structure of the additional unit and the necessary accuracy. Models based on the Galerkin method are quite common, so the authors \cite{Sirignano_2018} proposed the structure of the model so that, with an increase in the dimension of the problem, the quality of the solution remains acceptable. Their model looks interesting, combines many breakthrough deep learning approaches, but in view of this, the speed of learning is very low. The authors themselves in their work provide an assessment of the training time and the necessary capacities for this - it takes an order of magnitude more time on a conventional personal computer than classical approaches require, but the main goal is high-dimensional tasks, where the algorithm really showed good quality. All approaches proposed and considered below can be divided into 2 groups: 
\begin{itemize}
	\item Embed in the solution itself \cite{Lagaris_1998} \cite{liu2019solving} \cite{cao2016locally}
	\item Consider a conditional problem solved by the Lagrange method. In the learning process, the model learns not only to solve the equation itself, but is also fined for not satisfying the boundary conditions \cite{Pun_2019}
\end{itemize}
Each group has its own characteristics, so for methods from the first group, the high quality of the solution is characteristic, but the difficulty of drawing up the presentation of the solution is high. The second group is characterized by a not very high quality solution, especially at the borders, however, with sufficient training time and properly selected regularization, this problem is solved, but the plus is the ease of implementation.

\subsubsection{Examples of ODE}
\begin{equation*}
	\dfrac{d y}{d x} = sin(x), \quad y(0) = -1
\end{equation*}

\begin{tabular}{c c c}
 Method & Parameters num & Accuracy \\
 FDM & 10  & $7.82 10^{-3}$  \\
 FDM & 25  & $2.88 10^{-3}$  \\
 FDM & 50  & $1.44 10^{-3}$  \\
 FDM & 100  & $0.69 10^{-3}$ \\
  FDM & 200  & $0.34 10^{-3}$ \\
 ANN & 8  & $0.298 10^{-3}$  \\
 ANN & 10  & $0.111 10^{-3}$ \\
 ANN & 20  & $0.0105 10^{-3}$  \\
ANN & 50  & $0.00413 10^{-3}$
\end{tabular}