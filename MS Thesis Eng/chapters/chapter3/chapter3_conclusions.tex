\section*{Conclustions}
In this chapter, 2 problems were solved, the linear elasticity equation, which describes the deformations and stresses that arise under certain actions on the region, and the Stokes problem, which describes the fluid flow in a narrow channel. Both problems are described by systems of partial differential equations, linear and nonlinear. For each task, several different configurations of neural networks were considered. All networks were successfully trained, the quality and result of which greatly depends on the number of parameters, as well as their order. An important conclusion can be considered the fact that it makes no sense to use an arbitrary configuration, which will often be too exhaustive, and the required training time is too long. We could dwell on this conclusion, however, the treatability of the result remains another important issue, therefore, in Chapter 2, we considered possible approaches that would consider neural networks as an expansion of functions in some series. For this, 2 specific series, the Fourier series and the Chebyshev series are considered in sufficient detail. So, substituting the activation function of the cosine and the specially obtained criterion, one can obtain expansion in the Fourier series.