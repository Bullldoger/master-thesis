Не без некоторого колебания решился я избрать предметом настоящей лекции философию и идеал анархизма. Многие до сих пор еще думают, что анархизм есть не что иное, как ряд мечтаний о будущем или бессознательное стремление к разрушению всей существующей цивилизации. Этот предрассудок привит нам нашим воспитанием, и для его устранения необходимо более подробное обсуждение вопроса, чем то, которое возможно в одной лекции. В самом деле, давно ли — всего несколько лет тому назад — в парижских газетах пресерьезно утверждалось, что единственная философия анархизма — разрушение, а единственный его аргумент — насилие.
\par
Тем не менее об анархистах так много говорилось за последнее время, что некоторая часть публики стала наконец знакомиться с нашими теориями и обсуждать их, иногда даже давая себе труд подумать над ними; и в настоящую минуту мы можем считать, что одержали победу по крайней мере в одном пункте: теперь уже часто признают, что у анархиста есть некоторый идеал — идеал, который даже находят слишком высоким и прекрасным для общества, не состоящего из одних избранных.
\par
Но не будет ли, с моей стороны, слишком смелым говорить о философии в той области, где, по мнению наших критиков, нет ничего, кроме туманных видений отдаленного будущего? Может ли анархизм претендовать на философию, когда ее не признают за социализм вообще?